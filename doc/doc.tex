\documentclass[british]{article}
\usepackage[T1]{fontenc}
\usepackage[latin9]{inputenc}
\usepackage{geometry}
\geometry{verbose,lmargin=2cm,rmargin=2cm}
\usepackage{babel}
\usepackage{amsmath}
\usepackage{setspace}
\usepackage[unicode=true]
 {hyperref}

\makeatletter

\providecommand{\tabularnewline}{\\}

\usepackage{graphicx}
\usepackage{subfig}

\usepackage[newfloat]{minted}
\usepackage{xcolor}
%\usepackage{natbib}
\usepackage{tikz}
\usetikzlibrary{calc}
\usetikzlibrary{positioning}
\usetikzlibrary{arrows.meta}
\usepackage{caption}

\definecolor{magnolia}{rgb}{0.97, 0.96, 1.0}
\setminted{%
  bgcolor=magnolia,%
}

\newcommand\prv{bc}
\newcommand\m[1]{\mintinline{\prv}{#1}}
\newcommand\cm[1]{\mintinline{c}{#1}}

\newenvironment{code}{\captionsetup{type=listing}}{}
\SetupFloatingEnvironment{listing}{name=Source-Code}

\newcommand\scref[1]{Source-Code~\ref{code:#1}}

%\usepackage{subfiles}
\usepackage{standalone}

\usepackage{csquotes}
\usepackage[style=authoryear-ibid,backend=biber]{biblatex}
\addbibresource{references.bib}
%\usepackage[style=authoryear-ibid,backend=bibtex]{biblatex}
%\bibliography{references} 

\title{\texttt{\textsc{{[}prove{]}}}}

\makeatother

\begin{document}

\maketitle

\section{Introduction}

This document is work in progress. All source code as well as the most recent version of this document can be found on GitHub (see: \href{https://github.com/g-regex/prove/blob/main/doc/doc.pdf}{\texttt{https://github.com/g-regex/prove/blob/main/doc/doc.pdf}}). The purpose of this document to introduce
notation, vocabulary and some underlying principles in order to facilitate
efficient communication about this project. Furthermore this document
could serve as an introduction to the project, if anybody wants to
join.

\pagebreak{}

\section{Terminology}

Let us start with some basic definitions
\begin{itemize}
\item A \emph{statement} is a concatenation of one or more pair(s) of square
brackets enclosing another statement, a formula or an identifier.
\item A \emph{formula} is a concatenation of statements and formulators.
The empty string is a special case of a formula. Apart from this special
case every formula must contain at least one statement and at least
one formulator.
\item A \emph{formulator }is a sequence of legal (see later definition)
characters. A formulator does not contain any square brackets but
is delimited by a closing square bracket or the beginning of the file
at its beginning and an opening square bracket or the end of the file
at its end.
\item An \emph{identifier} is a sequence of legal characters and is delimited
by an opening square bracket at its beginning and by a closing square
bracket at its end.
\item To refer to formulators or identifiers more generally, we will use
the term \emph{symbol }in this document.
\item To refer to statements or formulas more generally, we will use the
term \emph{expression }in this document.
\item To refer to a formula, which is not an equality and not an implication, the term \emph{ordinary formula} will be used throughout the document.
\item When we speak of a \emph{character} in this document, we mean a uppercase
or lowercase Latin letter, a number or a special character from a
list that may be extended. Opening and closing square brackets, the
literals used in the reserved formulators \m{=}, \m{=>} and the
negation \m{!} are not referred to as characters in this document.
\item In the current implementation and negated statement is still a statement
and not a formula. The \m{!} is currently not a formulator - probably
this has to be discussed.
\item The terms \emph{variable} and \emph{constant} can be used to refer
to an identifier. Which term is to be used depends on the context,
we are using it in (see later definition).
\item The term \emph{real constant} refers to a formula, comprised of exactly
one formulator and exactly one empty statement.
\end{itemize}
\pagebreak{}

\section{Syntax}

\subsection{EBNF}

The syntax of the \texttt{\textsc{{[}prove{]}}} language can conveniently
be described by means of the extended Backus-Naur form%\cite{EBNF Wikipedia page}
(EBNF), which is a meta syntax notation defined in the ISO/IEC 14977
%\cite{ISO/IEC 14977 standard} standard. Using an EBNF enables us
to recursively and unambiguously describe nested patterns by grouping
syntactic elements to so called $\langle\texttt{production}\rangle$s.
It must always be possible to `unpack' a production in a unique
and finite way, resulting in a (sequence of) \textit{terminal} symbols,
indicated by quotation marks. In order to correctly interpret the
\texttt{\textsc{{[}prove{]}}}-EBNF it is sufficient to understand
the meaning of the following meta characters:

\bigskip{}

\begin{doublespace}
\begin{tabular}{l|l}
$|$ & indicates `or'\tabularnewline
$(\dots)$ & groups productions and/or terminals together; only useful in combination
with $|$\tabularnewline
$[\dots]$ & indicates optionality\tabularnewline
$\{\dots\}$ & indicates zero or more repetitions\tabularnewline
\end{tabular}\bigskip{}

\end{doublespace}

Using this notation the \texttt{\textsc{{[}prove{]}}} language can
be described by the following EBNF:

%% LyX 2.3.5.2 created this file.  For more info, see http://www.lyx.org/.
%% Do not edit unless you really know what you are doing.
\documentclass[british]{article}
\usepackage[T1]{fontenc}
\usepackage[latin9]{inputenc}
\usepackage{geometry}
\geometry{verbose,lmargin=2cm,rmargin=2cm}
\usepackage{amsmath}

\makeatletter
%%%%%%%%%%%%%%%%%%%%%%%%%%%%%% User specified LaTeX commands.
\usepackage{csquotes}

\makeatother

\usepackage{babel}
\begin{document}
\begin{align*}
\langle\texttt{expression}\rangle & =\langle\texttt{statement}\rangle\Big|\langle\texttt{formula}\rangle\\
\langle\texttt{formula}\rangle & =\langle\texttt{formulator\ensuremath{\rangle}}\Bigg|\bigg\{\langle\texttt{formulator}\rangle\langle\texttt{statement}\rangle\bigg\}\\
\langle\texttt{statement}\rangle & =\textrm{`['}\langle\texttt{expression}\rangle\textrm{{`]'}}\Big\{\textrm{`['}\langle\texttt{expression}\rangle\textrm{`]'}\Big\}\\
\langle\texttt{formulator\ensuremath{\rangle}} & =\textrm{`='\ensuremath{\Big|}`=>'\ensuremath{\Big|}}\Big\{\langle\texttt{character}\rangle\Big\}\\
\langle\texttt{character\ensuremath{\rangle}} & =\textrm{`a'}\Big|\dots\Big|\textrm{`z'}\Big|\textrm{`A'}\Big|\dots\Big|\textrm{`Z'}\Big|\textrm{`0'}\Big|\dots\Big|\textrm{`9'}\Big|\textrm{`+'\ensuremath{\Big|\textrm{`-'}\Big|\textrm{`/'\ensuremath{\Big|\textrm{`*'}\Big|\textrm{`\%'\ensuremath{\Big|\textrm{`\textasciicircum'}\Big|\textrm{`\&'}\Big|\textrm{`.'}\Big|\textrm{`?'}\Big|}`!'\ensuremath{\Big|\textrm{`:'}}}}}}}\Big|\textrm{`\_'}
\end{align*}

\end{document}


\medskip{}

Every EBNF needs to have an initial production as a starting point.
It is convention that this distinguished production is to be listed
first, so in this case, $\langle\texttt{expr}\rangle$ is the initial
production. As one might notice a valid \texttt{{[}PROVE{]}} file
might contain either a statement or a formula. This design choice
has been made in order to account for axioms and theorems that might
be given in the human readable form of the \texttt{{[}PROVE{]}} language.
Proofs on the other hand will be given in the form of a formula, having
only \m{=>} formulators at its outermost level. We could easily think
of other formulas, which will not satisfy this requirement and thereby
neither constitute an axiom/theorem nor a proof. A \texttt{{[}PROVE{]}}
file containing such a formula would be syntactically correct but
would violate semantic rules, which will be established later. \pagebreak{}

\subsubsection{Examples}

Here are some examples of productions and some motivations for the
design of the syntax. We will work our way up from bottom to top through
the EBNF.
\begin{itemize}
\item A $\langle\texttt{character}\rangle$ can be any letter, number or
one of the 10 special characters as indicated in the EBNF\@. Examples
of valid $\langle\texttt{character}\rangle$s are \m{a}, \m{T},
\m{7}, \m{&}.
\item A $\langle\texttt{symbol}\rangle$ is a sequence of one or more characters.
Examples of valid $\langle\texttt{symbols}\rangle$s are: \m{a7we},
\m{T}, \m{76}, \m{&q}. Note that \m{T} is a $\langle\texttt{symbol}\rangle$
and a $\langle\texttt{character}\rangle$ at the same time. This does
not mean that the grammar is ambiguous. A single \m{T} will always
be interpreted in this way, but it depends on the context, whether
we are interested in the fact of \m{T} being a $\langle\texttt{symbol}\rangle$
or \m{T} being a $\langle\texttt{character}\rangle$.
\item A $\langle\texttt{formulator}\rangle$ is either a $\langle\texttt{symbol}\rangle$
or any one of the terminals (fixed, not \textquotedbl unpack-able\textquotedbl{}
elements of an EBNF) \m{=} or \m{=>}. Examples of valid $\langle\texttt{formulator}\rangle$s
are \m{a7we}, \m{T}, \m{=}, \m{=>}. Note that \m{=T} is not a
valid $\langle\texttt{formulator}\rangle$. Seeing that \m{T} can
be a $\langle\texttt{formulator}\rangle$ (comprised of a $\langle\texttt{symbol}\rangle$)
in one case and just a $\langle\texttt{symbol}\rangle$ (not packed
inside a $\langle\texttt{formulator}\rangle$) in another case, one
might worry about ambiguity. However, having a close look at the EBNF,
we see that $\langle\texttt{formulator}\rangle$s only occur in between
$\langle\texttt{statement}\rangle$s and $\langle\texttt{string}\rangle$s
only within $\langle\texttt{statement}\rangle$s. The user of the
language has great freedom in naming formulators and identifiers (TODO:
define formulators and identifiers in this document, i.e.\ not the
productions, but what they represent), so it is up to the user to
choose them wisely. It could for example be confusing to name a formulator
\m{7}, but \m{add7} makes intuitively sense.
\item A $\langle\texttt{statement}\rangle$ is a concatenation of one or
more pairs of square brackets each containing either a $\langle\texttt{formula}\rangle$,
a $\langle\texttt{symbol}\rangle$ or a $\langle\texttt{statement}\rangle$.
Examples of valid $\langle\texttt{statement}\rangle$s are \m{[]},
\m{[a][b][[a]in[b]]}, \m{[[]]}, \m{[[a]or[b]]}.
\item A $\langle\texttt{formula}\rangle$ is comprised of at least one $\langle\texttt{statement}\rangle$
and at least one $\langle\texttt{formulator}\rangle$. The order of
occurrence does not matter. Examples of valid $\langle\texttt{formula}\rangle$e
are \m{[a]in[b]}, \m{[a]=[b][c]}, \m{[[a]a7we[b]]=>[c]}.
\end{itemize}
%\pagebreak{}

\section{Semantic}

When generating a couple of syntactically correct snippets of \texttt{\textsc{{[}prove{]}}}
code, one realises that not all code adhering to the rules of the
EBNF makes sense - at least not without introducing more conventions.

\medskip{}

Consider the following code:

\begin{minted}{\prv}
[a]and[b]=[c]=>[b][c]
\end{minted}

The use of \m{=} and \m{=>} is ambiguous in this context, since
\m{=} could refer to everything on its right hand side or just to
\m{[c]}. Similarly \m{=>} could refer to everything to its left
hand side or just to \m{[c]}. This problem could be addressed by
introducing an order of precedence (e.g.\ non reserved formulators,
\m{=}, \m{=>}). However, the implications of this approach should
be carefully considered. Precedence can easily be encoded in the EBNF,
without changing the set of valid code patterns. By doing so we could
elegantly encode a part of the language's meaning in its syntactic
structure. As tempting as this idea might be, I can see a negative
side effect coming along with this approach. Introducing precedence
between formulators will set brackets \textit{implicitly. }Since the
visibility and meaning of identifiers (variables/constants) depends
on the depth of nesting of statements (i.e.\ on brackets), the approach
of precedence might lead to counter-intuitive code. Therefore I recommend
against the precedence approach.

\pagebreak{}

Instead I suggest the following semantic rule, which will address
this issue:
\begin{itemize}
\item \m{=}, \m{=>} and other formulators must not be mixed
\end{itemize}
\medskip{}

Consider the following code:

\begin{minted}{\prv}
[a][b]=>[c]
\end{minted}

Translated to English this means `Let a and b be identifiers, then
we have an identifier c'. We do not know, where c comes from, since
it was not introduced before. Therefore I suggest the following rule:
\begin{itemize}
\item New identifiers must only be introduced in a part of a formula, which
is reserved for assumptions.
\end{itemize}
\medskip{}

To determine, which statements within a formula are assumptions and
which statements are conclusions, I'd like to put the following three
rules up for discussion:
\begin{itemize}
\item If the formula is an implication (i.e., all formulators are \m{=>}),
then every statement occurring before the first formulator is an assumption.
\item Every statement in an equality (i.e., all formulators are \m{=})
is an assumption, from which all other statements can be concluded.
(Remark: An equality has all the properties a two-sided implication
would have, but in addition to that in also allows substituting one
statement with another)
\item If a statement is an assumption, all statements contained in it are
assumptions as well.
\end{itemize}
\medskip{}

It might also be convenient to restrict the form an equality can have.
Especially with respect to a straight forward implementation, it would
make sense to only let single statements (i.e.\ only one pair of square
brackets at the outermost level) to be equal to each other. Therefore
I suggest the following rules:
\begin{itemize}
\item Equalities are only such formulas, which are a concatenation of single
statements and \m{=}, where no single statement is next to another
single statement and no \m{=} is next to another \m{=}
\item An equality must start and end with a single statement.
\end{itemize}
\medskip{}
Further I suggest the following rule for implications for obvious
reasons:
\begin{itemize}
\item \m{=>} must not appear at the end of a $\langle\texttt{statement}\rangle$
\end{itemize}
\pagebreak{}

\section{Representation}

Let us have a look at the following statement:

\begin{code}
\captionof{listing}{Example 1}
\label{code:first}
\inputminted[linenos]{\prv}{examples/exmp1.prove}
\end{code}

An intuitive approach for representing the structure of this proof
is by means of a graph (in particular a special planar tree).

\begin{figure}[!h]
\caption{Representation of \scref{first}}
\label{russell_graph}
\centering

\subfloat[\centering Legend]{
	\includegraphics[width=\textwidth, height=10cm, keepaspectratio]{tikz/legend.pdf}
}
\subfloat[\centering Graph of \scref{first}]{
	\includegraphics[width=\textwidth, height=10cm, keepaspectratio]{tikz/exmp1.pdf}
}

\end{figure}

\pagebreak{}

\section{Verification}

\subsection{Variables and constants}

TODO
%%Identifiers can be either variables, constants or unknown depending
%%on the context (i.e.\ the node from whose perspective we intend to
%%make use of that identifier). In the following, we will say that a
%%node is \emph{below} another node, when it is its child or below its
%%child. Further we will say that a node is a \emph{younger sibling}
%%of another node, if it is right of that node or right of a younger
%%sibling of that node. We can now define the following:
%%\begin{itemize}
%%\item An identifier is a constant for all nodes, which are below the younger
%%siblings of the parent node of the identifier.
%%\item An identifier is a variable for all nodes, which are below the younger
%%siblings of the parent node of the parent node of the identifier.
%%\item An identifier is unknown for every other node.
%%\end{itemize}

\subsection{Negation}

TODO
\pagebreak{}

\section{Output}

TODO
%\begin{code}
%\captionof{listing}{\texttt{[prove]} output after processing \scref{russell}}
%\label{code:russell_output}
%\begin{minted}[linenos]{\prv}
%
%\end{minted}
%\end{code}
%
%Currently the \texttt{\textsc{{[}prove{]}}} software produces hardly
%more than debugging output. It can be very helpful though for understanding
%the process of verification. To refer to the nodes of the graph, the
%nodes are numbered in pre-order traversal %\cite{pre-order traversal}
%as illustrated in MISSING The corresponding
%output (\scref{russell_output} {[}some line breaks have been added
%to improve readibility{]}) can be understood as follows:
%\begin{itemize}
%\item When verifying a node, \texttt{\textsc{{[}prove{]}}} prints the number
%of the node within curly braces (e.g. \m{{83}}).
%\item The number of every reachable node that the node to be verified is
%checked against is printed within angle brackets (e.g. \m{<76>}).
%\item If the verification was successful, a hash is added to the number
%of the reachable node (e.g. \texttt{<64\#>}).
%\item If verification of the current node is not necessary, because the
%VRFD flag is set, a star is printed before the node number (e.g. \m{*{67}}
%\end{itemize}
%Currently no verification steps are skipped and nodes are often verified
%multiple times. This is desired at the current stage for debugging
%purposes and to illustrate the underlying verification mechanism.
%
%Most semantic rules are also still not checked and some of them still
%have to be defined. One very important issue, which I would like to
%discuss, is how identifiers are allowed to be introduced. For the
%verification of this proof I had to relax a few restrictions that
%I have originally imposed on this issue, but without further elaboration
%of this, I suggest a proper discussion of the matter.

%\begin{thebibliography}{1}
%\bibitem{=00005Bprove=00005D GitHub repository}\href{https://github.com/g-regex/prove}{https://github.com/g-regex/prove}
%
%\bibitem{latest}\href{https://github.com/g-regex/prove/blob/master/doc/doc.pdf}{https://github.com/g-regex/prove/blob/master/doc/doc.pdf}
%
%\bibitem{EBNF Wikipedia page}\href{https://en.wikipedia.org/wiki/Extended_Backus\%E2\%80\%93Naur_form}{https://en.wikipedia.org/wiki/Extended\_Backus\%E2\%80\%93Naur\_form}
%
%\bibitem{ISO/IEC 14977 standard}\href{https://www.cl.cam.ac.uk/~mgk25/iso-14977.pdf}{https://www.cl.cam.ac.uk/$\sim$mgk25/iso-14977.pdf}
%
%\bibitem{BNF Wikipedia page}\href{https://en.wikipedia.org/wiki/Backus\%E2\%80\%93Naur_Form}{https://en.wikipedia.org/wiki/Backus\%E2\%80\%93Naur\_Form}
%
%\bibitem{pre-order traversal}\href{https://en.wikipedia.org/wiki/Tree_traversal\#Pre-order_(NLR)}{https://en.wikipedia.org/wiki/Tree\_traversal\#Pre-order\_(NLR)}
%\end{thebibliography}

\printbibliography

\end{document}
